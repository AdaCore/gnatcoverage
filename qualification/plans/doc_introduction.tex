\section{Referenced documents}
\paragraph*{CAST Paper 17} \ \\
Certification Authorities Paper 17: Structural coverage of object code
\paragraph*{DO-178B and ED-12B} \ \\
EUROCAE: Software Considerations in Airborne Systems and Equipment Certification.
\paragraph*{DO-178C and ED-12C} \ \\
EUROCAE: Software Considerations in Airborne Systems and Equipment Certification.
\paragraph*{\xcov UG} \ \\
AdaCOre: \xcov User Guide

\section{Definitions list}

\paragraph*{\xcov} \ \\

\paragraph*{Decision Coverage(\dc)} \ \\

\paragraph*{Modified Decision Condition Coverage (\mcdc)} \ \\

\paragraph*{Object Branch Coverage (\obc)} \ \\

\paragraph*{Object Instruction Coverage (\oic)} \ \\

\paragraph*{Statement Coverage (\stc)} \ \\

\paragraph*{Tool Operational Requiremet (TOR)} \ \\

\section{Organization and roles}
Several parties are involved in the development, verification and qualification processes for \xcov. The interested parties - along with their responsibilities - are:
\begin{itemize}
\item \textbf{\xcov Development Team} contributes to the development of \xcov, including requirements specification, implementation, test cases development and test execution. This team is also in charge of the configuration management of the artifacts it produces.
\item \textbf{\xcov Qualification Team} is responsible for the infrastructure
supporting the qualification process of \xcov. The Qualification Team supports the development team. This team is also in charge of the configuration management of the artifacts it produces.
\item \textbf{\xcov Quality Assurance Team} is a project-independent team responsible to ascertain the expected processes have been put in place. The Quality Assurance Team is granted the authority to require specific activities to be performed by the \xcov Development and Qualification Teams. This team is also in charge of the configuration management of the artifacts it produces (mostly Quality Assurance reports).
\item \textbf{\xcov Users} are expected to perform the activities identified in
section \ref{sec:user-act}.
\end{itemize}
