\documentclass {report}
\usepackage{couverture}
\usepackage{color}
\definecolor{light-gray}{gray}{0.85}
\usepackage{listings}
\lstset{backgroundcolor=\color{light-gray}}

\begin{document}
\title{\huge xCov\\Software Quality Assurance Plan\\ \ \\ \large \textbf{Document Version 1.0}}

\maketitle
\tableofcontents

\chapter{Document introduction}

\section{Document purpose}
The intent of the Software Quality Assurance Plan is to describe the process to ascertain the correct application of the Tool Qualification Plan and of the Software Configuration Management Plans.

\section{Authors}
\begin{tabular}{|c|c|c|}
\hline
\textbf{Name} & \textbf{Company} & \textbf{Email} \\ \hline
Matteo Bordin & AdaCore & bordin@adacore.com \\ \hline
Olivier Hainque & AdaCore & hainque@adacore.com \\ \hline
\end{tabular}

\section{Major revision history}
The evolution of this document is automatically tracked by the configuration management system. Here we just provide the major revision history.
\ \\ \\
\begin{tabular}{|c|c|c|}
\hline
\textbf{Version} & \textbf{Date} & \textbf{Comment} \\ \hline
 &  &  \\ \hline
\end{tabular}

\newcommand{\erts}{[ERST2010]\space}
\newcommand{\adaeurope}{[AE2009]\space}
\newcommand{\castseventeen}{[CAST17]\space}

\section{Referenced documents}
\paragraph*{\adaeurope} \ \\
Bordin et al.: Couverture: An Innovative Open Framework for Coverage Analysis of Safety Critical Applications - Ada User Journal, December 2009.
%\paragraph*{\castseventeen} \ \\
%Certification Authorities Paper 17: Structural coverage of object code
\paragraph*{[DO-178B] and [ED-12B]} \ \\
EUROCAE: Software Considerations in Airborne Systems and Equipment Certification.
\paragraph*{[DO-178C] and [ED-12C]} \ \\
EUROCAE: Software Considerations in Airborne Systems and Equipment Certification.
\paragraph*{\erts} \ \\
Bordin et al: Couverture: An Innovative and Open Coverage Analysis Framework for Safety-Critical Applications- ERTS2 2010
\paragraph*{[\xcov{} UG]}
AdaCore: \xcov{} Fundamentals and Users Guide

\section{Definitions list}

\paragraph*{Coverage exception} \ \\

\paragraph*{Decision Coverage(\dc)} \ \\

\paragraph*{Modified Decision Condition Coverage (\mcdc)} \ \\

\paragraph*{Object Branch Coverage (\obc)} \ \\

\paragraph*{Object Instruction Coverage (\oic)} \ \\

\paragraph*{Statement Coverage (\stc)} \ \\

\paragraph*{Tool Operational Requiremet (TOR)} \ \\

\section{Organization and roles}
Several parties are involved in the development, verification and qualification processes for \xcov{}. The interested parties - along with their responsibilities - are:
\begin{itemize}
\item \textbf{\xcov{} Development Team} contributes to the development of \xcov{}, including requirements specification, implementation, test cases development and test execution. This team is also in charge of the configuration management of the artifacts it produces.
\item \textbf{\xcov{} Qualification Team} is responsible for the infrastructure
supporting the qualification process of \xcov{}. The Qualification Team supports the development team. This team is also in charge of the configuration management of the artifacts it produces.
\item \textbf{\xcov{} Quality Assurance Team} is a project-independent team responsible to ascertain the expected processes have been put in place. The Quality Assurance Team is granted the authority to require specific activities to be performed by the \xcov{} Development and Qualification Teams. This team is also in charge of the configuration management of the artifacts it produces (mostly Quality Assurance reports).
\item \textbf{\xcov{} Users} are expected to perform the activities identified in
section \ref{sec:user-act}.
\end{itemize}


\chapter{Quality assurance activities}
This sections contains a description of the Quality Assurance activities in terms of objective and output.

\section{Reading of \xcov qualification material: plans}
\begin{itemize}
\item \textbf{objectives:} to assess the compliance with qualification objectives;
\item \textbf{output:} QA Reading report (qa/DDMMYYYY/qa\_plans.doc).
\end{itemize}

\section{Inspection of Tool Operational Requirements (by sampling)}
\begin{itemize}
\item \textbf{objectives:}
\begin{itemize}
\item check the accuracy and completeness with the targeted language feature and coverage metric;
\item check the compliance with the development standard.
\end{itemize}
\item \textbf{output:} QA inspection report (qa/DDMMYYYY/qa\_tor.doc).
\end{itemize}

\section{Inspection of Test Cases (by sampling)}
\begin{itemize}
\item \textbf{objectives:}
\begin{itemize}
\item check the representativeness of used programming language constructs for each considered coverage metrics;
\item check the compliance with the development standard.
\end{itemize}
\item \textbf{output:} QA inspection report (qa/DDMMYYYY/qa\_tor.doc, the same file used for the review of Tool Operational Requirements).
\end{itemize}

\section{Inspection of test execution results}
\begin{itemize}
\item \textbf{objectives:}
\begin{itemize}
\item Check the results of test execution
\item In the case tests failed, it is necessary to investigate whether the source of error is:
\begin{itemize}
\item A misbehaviour of the infrastructure used to run tests and compare actual results to expected results: in this case, the Qualification Team is in charge
of reporting and fixing the problem.
\item A bug in the \xcov implementation: in this case, the Development
Team is in charge of reporting and fixing the problem.
\item A reasonable limitation of the tool: in this case, the Qualification
Team is in charge of reporting and justifying the problem as part of \xcov known constraints and open problems.
\end{itemize}
\end{itemize}
\item \textbf{output:} QA inspection report (qa/DDMMYYYY/qa\_test\_execution.doc).
\end{itemize}

\section{Tool conformity review}
The conformity review takes in input a specific packaged and qualifiable release of \xcov.
\begin{itemize}
\item \textbf{objectives:}
\begin{itemize}
\item review the delivery file in the scope of the precise packaged and qualifiable release of \xcov
\item consistency within the Qualification Material;
\item finalize discussion on open problems (if any).
\end{itemize}
\item \textbf{output:} QA inspection report qa/<RELEASE>/DDMMYYYY/ qa\_conformity.doc).
\end{itemize}


\section{Responsibility}
The responsibility for the Quality Assurance Process belongs to the Quality Assurance Team.

\end{document}
