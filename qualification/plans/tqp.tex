\documentclass {report}
\usepackage{couverture}
\usepackage{color}
\definecolor{light-gray}{gray}{0.85}
\usepackage{listings}
\lstset{backgroundcolor=\color{light-gray}}

\begin{document}
\title{\huge xCov\\Tool Qualification Plan\\ \ \\ \large \textbf{Document Version 1.0}}

\maketitle
\tableofcontents

\chapter{Document introduction}

\section{Document purpose}
The purpose of this document is to describe the applicable processes to qualify \xcov in a DO-178B and DO-178C context.

\section{Authors}
\begin{tabular}{|c|c|c|}
\hline
\textbf{Name} & \textbf{Company} & \textbf{Email} \\ \hline
Matteo Bordin & AdaCore & bordin@adacore.com \\ \hline
Olivier Hainque & AdaCore & hainque@adacore.com \\ \hline
\end{tabular}

\section{Major revision history}
The evolution of this document is automatically tracked by the configuration management system. Here we just provide the major revision history.
\ \\ \\
\begin{tabular}{|c|c|c|}
\hline
\textbf{Version} & \textbf{Date} & \textbf{Comment} \\ \hline
 &  &  \\ \hline
\end{tabular}

\newcommand{\erts}{[ERST2010]\space}
\newcommand{\adaeurope}{[AE2009]\space}
\newcommand{\castseventeen}{[CAST17]\space}

\section{Referenced documents}
\paragraph*{\adaeurope} \ \\
Bordin et al.: Couverture: An Innovative Open Framework for Coverage Analysis of Safety Critical Applications - Ada User Journal, December 2009.
%\paragraph*{\castseventeen} \ \\
%Certification Authorities Paper 17: Structural coverage of object code
\paragraph*{[DO-178B] and [ED-12B]} \ \\
EUROCAE: Software Considerations in Airborne Systems and Equipment Certification.
\paragraph*{[DO-178C] and [ED-12C]} \ \\
EUROCAE: Software Considerations in Airborne Systems and Equipment Certification.
\paragraph*{\erts} \ \\
Bordin et al: Couverture: An Innovative and Open Coverage Analysis Framework for Safety-Critical Applications- ERTS2 2010
\paragraph*{[\xcov{} UG]}
AdaCore: \xcov{} Fundamentals and Users Guide

\section{Definitions list}

\paragraph*{Coverage exception} \ \\

\paragraph*{Decision Coverage(\dc)} \ \\

\paragraph*{Modified Decision Condition Coverage (\mcdc)} \ \\

\paragraph*{Object Branch Coverage (\obc)} \ \\

\paragraph*{Object Instruction Coverage (\oic)} \ \\

\paragraph*{Statement Coverage (\stc)} \ \\

\paragraph*{Tool Operational Requiremet (TOR)} \ \\

\section{Organization and roles}
Several parties are involved in the development, verification and qualification processes for \xcov{}. The interested parties - along with their responsibilities - are:
\begin{itemize}
\item \textbf{\xcov{} Development Team} contributes to the development of \xcov{}, including requirements specification, implementation, test cases development and test execution. This team is also in charge of the configuration management of the artifacts it produces.
\item \textbf{\xcov{} Qualification Team} is responsible for the infrastructure
supporting the qualification process of \xcov{}. The Qualification Team supports the development team. This team is also in charge of the configuration management of the artifacts it produces.
\item \textbf{\xcov{} Quality Assurance Team} is a project-independent team responsible to ascertain the expected processes have been put in place. The Quality Assurance Team is granted the authority to require specific activities to be performed by the \xcov{} Development and Qualification Teams. This team is also in charge of the configuration management of the artifacts it produces (mostly Quality Assurance reports).
\item \textbf{\xcov{} Users} are expected to perform the activities identified in
section \ref{sec:user-act}.
\end{itemize}


\chapter{Tool overview}
\section {Sought certification credit}
DO-178 requires to measure test coverage of software structure as part of the Software Verification Process. Measure of test coverage is mandatory for level A, B, C. Level A requires to achieve at least \mcdc, \dc and \stc (table A7 objective 5-7); level B requires to achieve at least \dc and \stc (table A7 objective 6-7) and level C requires to achieve at least \stc (table A7 objective 7). In the case of level A software, and if the compilers generates un-traceable object code, additional verification may be required on object code. CAST Paper 17 suggests at least two possible activities to achieve this additional objective: to measure test coverage of object code and to develop a traceability study on a representative set of source code.

\xcov aims at automating the measurement of activities above. In the case of level A software, \xcov also aims at automating the measurement of test coverage of object code if this strategy has been chosen to cope with un-traceable object code.

\xcov is qualified as a \emph{verification tool} or, in DO-178C terms, with a Tool Qualification Level 3 (TQL3). As such, and as long as \xcov is used following its qualified interface (see section \ref{sec:qual-interface}), applicants may use \xcov output as certification evidence and certification authorities shall consider \xcov output \emph{as good as} the output of a manual activity.

\section{Qualified interface}
\label{sec:qual-interface}

\chapter{Qualification data}

\section{Tool development and verification standards}
\subsection{Tool Operational Requirements}
A physical folder on the repository is associated to each TOR. The name of the folder is the id of the TOR. All qualification artifacts derived from a TOR are stored as either files or folders contained in the folder of the parent TOR. The TOR textual specification is contained in a file always named \texttt{req.txt}.

\subsection{Test cases}
Explicit description of test cases (when necessary) is contained in the \texttt{req.txt} of the parent TOR under the section "Testing strategy". Source code for test cases is contained in the \texttt{src} sub-folder and source files do \emph{not} start with the \texttt{test\_} prefix. 

\subsection{Tests source code}
Source code for tests is contained in the \texttt{src} sub-folder; source files start with the \texttt{test\_} prefix. 

\section{Production of qualification data}
\subsection{Environment equivalence}
\subsection{Tool Operational Requirements}
\subsection{Testing strategy}
\subsection{Test cases}
\subsection{Tests}
\subsection{Verification results analysis}
\subsection{Specific constraints and open problems}
\subsection{Delivery file}

\chapter{User activities}
\label{sec:user-act}

\end{document}
